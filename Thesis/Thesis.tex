\documentclass[]{article}
\usepackage{lmodern}
\usepackage{amssymb,amsmath}
\usepackage{ifxetex,ifluatex}
\usepackage{fixltx2e} % provides \textsubscript
\ifnum 0\ifxetex 1\fi\ifluatex 1\fi=0 % if pdftex
  \usepackage[T1]{fontenc}
  \usepackage[utf8]{inputenc}
\else % if luatex or xelatex
  \ifxetex
    \usepackage{mathspec}
  \else
    \usepackage{fontspec}
  \fi
  \defaultfontfeatures{Ligatures=TeX,Scale=MatchLowercase}
\fi
% use upquote if available, for straight quotes in verbatim environments
\IfFileExists{upquote.sty}{\usepackage{upquote}}{}
% use microtype if available
\IfFileExists{microtype.sty}{%
\usepackage[]{microtype}
\UseMicrotypeSet[protrusion]{basicmath} % disable protrusion for tt fonts
}{}
\PassOptionsToPackage{hyphens}{url} % url is loaded by hyperref
\usepackage[unicode=true]{hyperref}
\hypersetup{
            pdftitle={Trilateration Index - Masters Thesis},
            pdfauthor={Chip Lynch},
            pdfborder={0 0 0},
            breaklinks=true}
\urlstyle{same}  % don't use monospace font for urls
\usepackage[margin=1in]{geometry}
\usepackage{graphicx,grffile}
\makeatletter
\def\maxwidth{\ifdim\Gin@nat@width>\linewidth\linewidth\else\Gin@nat@width\fi}
\def\maxheight{\ifdim\Gin@nat@height>\textheight\textheight\else\Gin@nat@height\fi}
\makeatother
% Scale images if necessary, so that they will not overflow the page
% margins by default, and it is still possible to overwrite the defaults
% using explicit options in \includegraphics[width, height, ...]{}
\setkeys{Gin}{width=\maxwidth,height=\maxheight,keepaspectratio}
\IfFileExists{parskip.sty}{%
\usepackage{parskip}
}{% else
\setlength{\parindent}{0pt}
\setlength{\parskip}{6pt plus 2pt minus 1pt}
}
\setlength{\emergencystretch}{3em}  % prevent overfull lines
\providecommand{\tightlist}{%
  \setlength{\itemsep}{0pt}\setlength{\parskip}{0pt}}
\setcounter{secnumdepth}{0}
% Redefines (sub)paragraphs to behave more like sections
\ifx\paragraph\undefined\else
\let\oldparagraph\paragraph
\renewcommand{\paragraph}[1]{\oldparagraph{#1}\mbox{}}
\fi
\ifx\subparagraph\undefined\else
\let\oldsubparagraph\subparagraph
\renewcommand{\subparagraph}[1]{\oldsubparagraph{#1}\mbox{}}
\fi

% set default figure placement to htbp
\makeatletter
\def\fps@figure{htbp}
\makeatother


\title{Trilateration Index - Masters Thesis}
\author{Chip Lynch}
\providecommand{\institute}[1]{}
\institute{University of Louisville}
\date{6/24/2018}

\begin{document}
\maketitle

\section{Using Trilateration Distances as Geospatial Coordinates,
Indexes, and
Geohashes}\label{using-trilateration-distances-as-geospatial-coordinates-indexes-and-geohashes}

\subsubsection{Abstract}\label{abstract}

We present an alternative method for database indexing to improve query
performance for some distance related geospatial queries base on storing
distances from three fixed points as in trilateration. This effectively
creates a coordinate system -- i.e.~an alternative to Latitude and
Longitude -- where the coordinates are the trilateration distances. We
explore this alternative coordinate system and the theoretical,
technical, and practical implications of using it.

Initial results are promising for some use cases. Nearest-neighbor logic
is both simplified (compared to R-Tree style indexing) and performant.
Trilateration (or, more generally, ``n-point lateration'') is applicable
to 2D, 3D, and higher dimension systems with minimal adaptation. The
system is easily extensible to common geospatial database types such as
lines and polygons. The concept of ``Bounding Bands'' (rather than
``Bounding Boxes'') is introduced for coordinate systems on spheres and
ellipsoids.

Potential applications to this approach extend to any distance-measured
space, and we touch on those (briefly, here, to constrain our scope).
For example, we consider applying the technique to Levenshtein distance,
and even facial recognition or systems where distances do not follow the
triangle inequality.

\newpage

\subsection{Trilateration Index -- General
Definition}\label{trilateration-index-general-definition}

Given an n-dimensional metric space \((M, d)\) (the universe of points
\(M\) in the space and a distance function \(d\) which respects the
triangle inequality), a typical point \(X\) in the coordinate system
will be described by coordinates \({x_1, x_2, …, x_n}\), which,
typically, represents the decomposition of a vector \(V\) from an
``origin'' point \(O: {0, 0, … ,0}\) to \(X\) into orthogonal vectors
\({0,x1}, {0,x2}, …, {0,xn}\) along each of the n dimensional axes of
the space.

The Trilateration of such a point requires \(n+1\) fixed points \(F_p\)
(\(p\) from \(1\) to \(n+1\)), no three of which occupy the same
\((n-1)\)-dimensional hyperplane. The Trilateration Coordinate for the
point \(X\) is then: \({t_1, t_2, …, t_{n+1}}\) where \(t_i\) is the
distance (according to \(d\)) from \(X\) to \(F_i\) (in units applicable
to the system).

\begin{center}\rule{0.5\linewidth}{0.5pt}\end{center}

\subsection{2-D Bounded Example}\label{d-bounded-example}

Consider a 2-dimensional grid -- a flattened map, a video game map, or
any mathematical \(x-y\) coordinate grid with boundaries. WOLOG in this
example consider the two-dimensional Euclidean space \(M=\mathbb{R^2}\)
and bounded by \(x,y\ \epsilon\  \{0..100\}\). Also, let us use the
standard Euclidean distance function for \(d\). This is, trivially, a
valid metric space.

Since the space has dimension \(n=2\), we need \(3\) fixed points
\(F_p\). While the Geospatial example on Earth has a specific
prescription for the fixed points, an arbitrary space does not. We
therefore prescribe the following construction for bounded spaces:

Construct a circle (hypersphere for other dimensions) with the largest
area inscribable in the space. In this example, that will be the circle
centered at \((50,50)\) with radius \(r=50\).

Select the point at which the circle touches the boundary at the first
dimension (for spaces with uneven boundary ratios, select the point at
which the circle touches the earliest boundary \(x_i\)). Such a point is
guaranteed to exist since the circle is largest (if it does not, then
the circle can be expanded since there is space between every point on
the circle and an axis, and it is not a largest possible circle).

From this point, create a regular \(n+1\)-gon (triangle here) which
touches the circle at \(n+1\) points. These are the points we will use
as \(F_p\). They are, by construction, not all co-linear (or in general
do not all exist on the same \(n\)-dimensional hyperplane) satisfying
our requirement {[}proof{]}.

The point \(y=0\), \(x=50\) is the first point of the equilateral
triangle. The slope of the triangle's line is \(tan(\frac{pi}{3})\), so
setting the equation of the circle:

\((x-50)^2+(y-50)^2=50^2\) equal to the lines:
\(y=tan(\frac{\pi}{3})(x-50)\) gives \(x=25(2+\sqrt{3})\) on the right
and \(y=tan(\frac{-\pi}{3})(x-50)\) gives \(x=-25(\sqrt{3}-2)\) on the
left, and of course the original \((0,50)\) point. Applying \(x\) to our
earlier equations for \(y\) we get a final set of three points:

\[F_1 = (x=50,y=0)\]

\[F_2 = (x=25(2+\sqrt{3}), y=tan(\frac{\pi}{3})((25(2+\sqrt{3}))-50)\]

\[F_3 = (x=-25(\sqrt{3}-2), y=tan(\frac{-\pi}{3})((-25(\sqrt{3}-2))-50) \]

\begin{center}\includegraphics{Thesis_files/figure-latex/trianglePlot-1} \end{center}

Remember, any three non-colinear points will do, but this construction
spaces them fairly evenly throughout the space, which may be beneficial
later* {[}Add section (reference) with discussions of precision and
examples where reference points are very near one another{]}.

The trilateration of any given point X in the space, now, is given by:

\[T(X) = {d(F_1, X), d(F_2, X), d(F_3,X)} \]

That is, the set of (three) distances \(d\) from \(X\) to \(F_1\),
\(F_2\), and \(F_3\) respectively.

\paragraph{10 Random Points}\label{random-points}

As a quick example of the trilateration calculations, we use a basic
collection of 10 data points:

\begin{verbatim}
##           x         y
## 1  58.52396 53.516719
## 2  43.73940 43.418684
## 3  57.28944  7.950161
## 4  35.32139 58.321179
## 5  86.12714 52.201894
## 6  41.08036 78.065907
## 7  51.14533 47.157734
## 8  15.42852 80.836340
## 9  85.13531 64.090063
## 10 99.60833 78.055071
\end{verbatim}

The trilateration of those points, that is, the three points
\(d_1, d_2, d_3 = {d(F_1, X), d(F_2, X), d(F_3,X)}\) are (next to the
respective \(x_n\)):

\begin{verbatim}
##           x         y       d1        d2       d3
## 1  58.52396 53.516719 54.19130 40.877779 56.10157
## 2  43.73940 43.418684 43.86772 58.768687 48.67639
## 3  57.28944  7.950161 10.78615 76.108693 83.99465
## 4  35.32139 58.321179 60.14002 60.331164 33.12763
## 5  86.12714 52.201894 63.48392 23.900246 82.63550
## 6  41.08036 78.065907 78.57382 52.310835 34.51806
## 7  51.14533 47.157734 47.17164 50.520446 52.44704
## 8  15.42852 80.836340 87.91872 78.091149 10.50105
## 9  85.13531 64.090063 73.08916 13.627535 79.19169
## 10 99.60833 78.055071 92.48557  7.008032 92.95982
\end{verbatim}

Note that we do not need to continue to store the original latitude and
longitude. We can convert the three \(d_n\) distances back to Latitude
and Longitude within some \(\epsilon\) based on the available precision.
Geospatial coordinates in Latitude and Longitude with six digits of
precision are accurate to within \(<1\ meter\), and 8 digits is accuract
to within \(<1\ centimeter\), although this varies based on the latitude
and longitude itself; latitudes closer to the equator are less accurate
than those at the poles. The distance values \(d_x\) are more
predictable, since they measure distances directly. While the units in
this sample are arbitrary, \(F(x)\) in a real geospatial example could
be in kilometers, so three decimal digits would precisely relate to
\(1\ meter\), and so on. This is one reason that we will later examine
using the trilateration values as an outright replacement for Longitude
and Latitide, and this feature is important when considering storage
requirements for this data in large real-world database applications.

For now, continuing with the example, those \(10\) points are shown here
in blue with the three reference points \(F_1, F_2, F_3\) in red:

\begin{center}\includegraphics{Thesis_files/figure-latex/plot10Points-1} \end{center}

To help understand the above values, the following chart shows the
distances for points \(x_2\) and \(x_10\) above. Specifically, the
distances \(d_1\) from point \(F_1\) are shown as arcs in red, the
distances \(d_2\) from point \(F_2\) in blue, and \(d_3\) from point
\(F_3\) in green.

\begin{center}\includegraphics{Thesis_files/figure-latex/distanceExamples-1} \end{center}

\paragraph{Use of trilateration as an index for
nearest-neighbor}\label{use-of-trilateration-as-an-index-for-nearest-neighbor}

One of our expected benefits of this approach is an improvement in
algorithms like nearest-neighbor search.

\subsubsection{Geospatial Example}\label{geospatial-example}

Applying this to real sample points; let the following be the initial
reference points on the globe:

Point 1: \(90.000000, 0.000000\) (The geographic north pole)

Point 2: \(38.260000, -85.760000\) (Louisville, KY on the Ohio River)

Point 3: \(-19.22000, 159.93000\) (Sandy Island, New Caledonia)

Optional Point 4: \(-9.42000, 46.33000\) (Aldabra)

Optional Point 5: \(-48.87000, -123.39000\) (Point Nemo)

Note that the reference points are defined precisely, as exact latitude
and longitude to stated decimals (all remaining decimal points are 0).
This is to avoid confusion, and why the derivation of the points is
immaterial (Point Nemo, for example is actually at a nearby location
requiring more than two digits of precision).

Only three points are required for trilateration (literally; thus the
``tri'' prefix of the term), but we include 5 points to explore the pros
and cons of n-fold geodistance indexing for higher values of n.

\subsection{Theoretical Discussion}\label{theoretical-discussion}

\subsubsection{Theoretical benefits:}\label{theoretical-benefits}

\textbf{Precision}: Queries are not constrained by precision choices
dictated by the index, as can be the case in Grid Indexes and similar
R-tree indexes. R-tree indexes improve upon naïve Grid Indexes in this
area, by allowing the data to dictate the size of individual grid
elements, and even Grid Indexes are normally tunable to specific data
requirements. Still, this involves analysis of the data ahead of time
for optimal sizing, and causes resistance to changes in the data.

\textbf{Distributed Computing}: Trilateration distances can be used as
hash values, compatible with distributed computing (I.e. MongoDB shards
or Teradata AMP Hashes).

\textbf{Geohashing}: Trilateration distances can be used as the basis
for Geohashes, which improve somewhat on Latitude/Longitude geohashes in
that distances between similar geohashes are more consistent in their
proximity.

\textbf{Bounding Bands}: The intersection of Bounding Bands create
effective metaphors to bounding boxes, without having to artificially
nest or constrain them, nor build them in advance.

\textbf{Readily Indexed (B-Tree compatible)}: Trilateration distances
can be stored in traditional B-Tree indexes, rather than R-tree indexes,
which can improve the sorting, merging, updating, and other functions
performed on the data.

\textbf{Fault Tolerant}: This coordinate system is somewhat
self-checking, in that many sets of coordinates that are individually
within the correct bounds, cannot be real, and can therefore be
identified as data quality issues. For example, a point cannot be 5
kilometers from the north pole (fixed point F1) and 5 kilometers from
Louisville, KY (fixed point F2) at the same time. A point stored with
those distances could be easily identified as invalid.

Theoretical shortcomings:

\textbf{Index Build Cost}: Up front calculation of each trilateration is
expensive, when translating from standard coordinates. Each point
requires three (at least) distance calculations from fixed points and
the sorting of the resulting three lists of distances. This results in
\texttt{O(n*logn)} just to set up the index.

*This could be mitigated by upgrading sensor devices and pushing the
calculations back to the data acquisition step, in much the way that
Latitude and Longitude are now trivial to calculate in practice by use
of GPS devices. Also, we briefly discuss how GPS direct measurements
(prior to converstion to Lat/Long) may be useful in constructing
trilateration values.

\textbf{Storage}: The storing of three distances (32- or 64- bits per
distance) is potentially a sizeable percent increase in storage
requirement from storing only Latitude/Longitude and some R-Tree or
similar index structure.

*Note that if the distances are stored instead of the Lat/Long, rather
than in addition to them, storage need not increase.

\textbf{Projection-Bound}: The up-front distance calculations means that
transforming from one spatial reference system (I.e. map projection --
geodetic -- get references to be specific) to another requires costly
recalculations bearing no benefit from the calculation. For example a
distance on a spherical projection of the earth between a given lat/long
combination will be different than the distance calculated on the earth
according to the standard WGS84 calculations).

*This said, we expect in most real-world situations, cross-geodetic
comparisons are rare.

\textbf{Difficult Bounding Band Intersection}: Bounding Bands intersect
in odd shapes, which, particularly on ellipsoids, but even on 2D grids,
are difficult to describe mathematically. Bounding boxes on the other
hand, while they distort on ellipsoids, are still easily understandable
as rectangles.

Figure 1 - An example problem with radio towers R1, R2, and R3, and
various receivers. Dashed lines represent the bounding bands with +/- a
small distance from a given receiver (center black circle)

Figure 2 - A close look at the intersection of three bounding bands
limiting an index search around a point with a search radius giving the
circle in A. Note that the area B is an intersection of two of the three
bands. Area C is the intersection of all three.

\section{Appendixy stuff}\label{appendixy-stuff}

Alternate Database Indexes

References (I need to finish reading and digesting these):

\url{http://ieeexplore.ieee.org/document/5437947/}

\url{https://prezi.com/chnisgybkshy/gps-trilateration/}

\url{https://www.maa.org/sites/default/files/pdf/cms_upload/Thompson07734.pdf}

\url{https://www.researchgate.net/publication/2689264_The_X-tree_An_Index_Structure_for_High-Dimensional_Data}

\url{http://repository.cmu.edu/cgi/viewcontent.cgi?article=1577\&context=compsci}

\url{https://link.springer.com/chapter/10.1007/10849171_83}

\url{http://ieeexplore.ieee.org.echo.louisville.edu/document/7830628/}

\url{https://link-springer-com.echo.louisville.edu/article/10.1007\%2Fs11222-013-9422-4}

\url{https://en.wikipedia.org/wiki/R-tree}

\url{https://boundlessgeo.com/2012/07/making-geography-faster/}

\url{http://ieeexplore.ieee.org.echo.louisville.edu/document/6045057/}

\url{http://www.sciencedirect.com.echo.louisville.edu/science/article/pii/S002002550900499X?_rdoc=1\&_fmt=high\&_origin=gateway\&_docanchor}=\&md5=b8429449ccfc9c30159a5f9aeaa92ffb\&ccp=y

\url{https://en.wikipedia.org/wiki/K-d_tree}

\url{http://citeseerx.ist.psu.edu/viewdoc/download?doi=10.1.1.219.7269\&rep=rep1\&type=pdf}

\url{http://www.scholarpedia.org/article/B-tree_and_UB-tree}

\end{document}
